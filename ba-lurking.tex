\documentclass{../cssheet}

%--------------------------------------------------------------------------------------------------------------
% Basic meta data
%--------------------------------------------------------------------------------------------------------------

\title{Lurking in Online-Communities am Beispiel eines Mathematik-Discord-Servers}
\author{Prof. Dr. Christian Spannagel}
\date{\today}
\hypersetup{%
    pdfauthor={\theauthor},%
    pdftitle={\thetitle},%
    pdfsubject={Thema Bachelorarbeit},%
    pdfkeywords={bachelor, phhd}
}

%--------------------------------------------------------------------------------------------------------------
% document
%--------------------------------------------------------------------------------------------------------------

\begin{document}

\vspace*{5mm}
\begin{center}
{\Large Thema für eine Bachelorarbeit}
\end{center}

\printtitle
\vspace*{1cm}

In Online-Communities beteiligt sich in der Regel nur ein kleiner Teil der Mitglieder aktiv an Diskussionen. Der weitaus größere Teil liest Beiträge, aber postet nichts (\emph{lurking}). Es ist eine große Herausforderung, diese Mitglieder aus der passiven Rolle in die Aktivität zu bringen.

In der Bachelorarbeit sollen ausgearbeitet werden, welche Gründe zu einer Lurking-Haltung führen und welche Maßnahmen ergriffen werden können, um mehr Mitglieder zu aktiven Beiträgen zu motivieren.

Mögliche Tätigkeiten sind:
\begin{itemize}
\item \textbf{Literaturrecherche:} Aufarbeitung der Literatur zu Lurking, den Motiven und möglichen Maßnahmen
\item \textbf{Analyse:} Untersuchung der Beteiligung in einer Mathematik-Community anhand der Server-Statistiken
\item \textbf{Empfehlungen:} Ableitung von Empfehlungen für das Community
\end{itemize}

\vspace*{10mm}

\printlicense

\printsocials



\end{document}
