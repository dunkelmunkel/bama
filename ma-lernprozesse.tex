\documentclass{../cssheet}

%--------------------------------------------------------------------------------------------------------------
% Basic meta data
%--------------------------------------------------------------------------------------------------------------

\title{Lernprozesse in der Vorbereitungsphase eines Inverted Classrooms}
\author{Prof. Dr. Christian Spannagel}
\date{\today}
\hypersetup{%
    pdfauthor={\theauthor},%
    pdftitle={\thetitle},%
    pdfsubject={Thema Masterarbeit},%
    pdfkeywords={master, phhd}
}

%--------------------------------------------------------------------------------------------------------------
% document
%--------------------------------------------------------------------------------------------------------------

\begin{document}

\vspace*{5mm}
\begin{center}
{\Large Thema für eine Masterarbeit}
\end{center}

\printtitle
\vspace*{1cm}

Im Inverted Classroom arbeiten sich Studierende selbstständig in Vorbereitung auf die Präenzphase in neue Themengebiete ein. Es ist bislang aber wenig bekannt darüber, wie die Studierenden die Vorbereitungsphase für sich selbst strukturieren und welche Aktivitäten sie in welchem Umfang durchführen.

In der Masterarbeit sollen die Lernaktivitäten der Studierenden in der Vorbereitungsphase eines Inverted Classrooms im Rahmen einer Mathematikveranstaltung untersucht werden. 

Mögliche Tätigkeiten sind:
\begin{itemize}
\item \textbf{Literaturrecherche:} Aufarbeitung der Literatur zu Lernaktivitäten in der Vorbereitungsphase eines Inverted Classrooms
\item  \textbf{Lern-Protokolle:} Gestaltung eines Lernprotokolls, das von den Studierenden wöchentlich ausgefüllt werden muss, und Analyse der ausgefüllten Protokolle
\item  \textbf{Interviews:} Durchführung von Interviews mit einzelnen Studierenden
\item \textbf{Empfehlungen:} Ableitung von Empfehlungen für die Durchführung eines Inverted Classrooms
\end{itemize}

\vspace*{10mm}

\printlicense

\printsocials



\end{document}
