\documentclass{../cssheet}

%--------------------------------------------------------------------------------------------------------------
% Basic meta data
%--------------------------------------------------------------------------------------------------------------

\title{Wirkmechanismen beim Flipped Classroom im Mathematikunterricht}
\author{Prof. Dr. Christian Spannagel}
\date{\today}
\hypersetup{%
    pdfauthor={\theauthor},%
    pdftitle={\thetitle},%
    pdfsubject={Thema Bachelorarbeit},%
    pdfkeywords={bachelor, phhd}
}

%--------------------------------------------------------------------------------------------------------------
% document
%--------------------------------------------------------------------------------------------------------------

\begin{document}

\vspace*{5mm}
\begin{center}
{\Large Thema für eine Bachelorarbeit}
\end{center}

\printtitle
\vspace*{1cm}

Die Wirksamkeit der Flipped Classroom Methode ist in der Forschung bereits umfassend untersucht worden. Weniger klar hingegen sind die zugrunde liegenden Wirkmechanis-men, die den Erfolg dieser Lehrmethode beeinflussen. Ziel der Arbeit ist es, den aktuellen Forschungsstand zu Moderator- und Mediatorvariablen im Zusammenhang mit der Flipped Classroom Methode zu ermitteln.

Mögliche Tätigkeiten sind:
\begin{itemize}
\item \textbf{Literaturrecherche:} Umfassende Aufarbeitung der vorhandenen Literatur zu den Moderator- und Mediatorvariablen im Kontext der Flipped Classroom Methode im Allgemeinen und im Mathematikunterricht im Speziellen
\item \textbf{Empfehlungen:} Ableitung von Empfehlungen für die Durchführung eines Inverted Classrooms
\end{itemize}

\vspace*{10mm}

\printlicense

\printsocials



\end{document}
