\documentclass{../cssheet}

%--------------------------------------------------------------------------------------------------------------
% Basic meta data
%--------------------------------------------------------------------------------------------------------------

\title{Gendersensible Mathematiklehre}
\author{Prof. Dr. Christian Spannagel}
\date{\today}
\hypersetup{%
    pdfauthor={\theauthor},%
    pdftitle={\thetitle},%
    pdfsubject={Thema Bachelorarbeit},%
    pdfkeywords={bachelor, phhd}
}

%--------------------------------------------------------------------------------------------------------------
% document
%--------------------------------------------------------------------------------------------------------------

\begin{document}

\vspace*{5mm}
\begin{center}
{\Large Thema für eine Bachelorarbeit}
\end{center}

\printtitle
\vspace*{1cm}

Mathematik gilt fälschlicherweise oft noch als „männliches Fach“. Frauen haben oftmals ein niedrigeres Selbstkonzept bzgl. Mathematik als Männer und entscheiden sich weni-ger häufig für MINT-Fächer bei ihrer Berufswahl. Aus diesem Grund sollte insbesondere die Mathematiklehre an Hochschulen so durchgeführt werden, dass Frauen nicht be-nachteiligt und Stereotype vermieden werden. In dieser Arbeit soll der Stand der For-schung zu gendersensibler Mathematiklehre zusammengestellt werden mit dem Ziel, Empfehlungen abzuleiten.

Mögliche Tätigkeiten sind:
\begin{itemize}
\item \textbf{Literaturrecherche:} Aufarbeitung der Literatur zu Mathematik und Gender im Allgemeinen und in der Hochschullehre im Speziellen
\item \textbf{Empfehlungen:} Ableitung von Empfehlungen für Lehrende zur Durchführung gendersensibler Ma-thematiklehre
\end{itemize}

\vspace*{10mm}

\printlicense

\printsocials



\end{document}
