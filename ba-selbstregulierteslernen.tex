\documentclass{../cssheet}

%--------------------------------------------------------------------------------------------------------------
% Basic meta data
%--------------------------------------------------------------------------------------------------------------

\title{Selbstreguliertes Lernen im Fach Mathematik}
\author{Prof. Dr. Christian Spannagel}
\date{\today}
\hypersetup{%
    pdfauthor={\theauthor},%
    pdftitle={\thetitle},%
    pdfsubject={Thema Bachelorarbeit},%
    pdfkeywords={bachelor, phhd}
}

%--------------------------------------------------------------------------------------------------------------
% document
%--------------------------------------------------------------------------------------------------------------

\begin{document}

\vspace*{5mm}
\begin{center}
{\Large Thema für eine Bachelorarbeit}
\end{center}

\printtitle
\vspace*{1cm}

Selbstreguliertes Lernen stellt hohe Anforderungen an Lernende. In Hochschul-Veranstaltungen im Fach Mathematik wird die Fähigkeit zum selbstregulierten Lernen oftmals vorausgesetzt, ohne die Studierenden auf dem Weg zu mehr Selbststeuerung zu begleiten. In der Arbeit soll der Stand der Forschung zum selbstregulierten Lernen im Fach Mathematik ermittelt werden mit dem Ziel, daraus Empfehlungen abzuleiten.

Mögliche Tätigkeiten sind:
\begin{itemize}
\item \textbf{Literaturrecherche:} Aufarbeitung der Literatur zum selbstregulierten Lernen im Allgemeinen und im Fach Mathematik im Speziellen
\item \textbf{Empfehlungen:} Ableitung von Empfehlungen für Lehrende, die Studierende beim Entwickeln von Selbstregulationskompetenzen begleiten wollen
\end{itemize}

\vspace*{10mm}

\printlicense

\printsocials



\end{document}
