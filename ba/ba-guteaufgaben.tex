\documentclass{../../cssheet}

%--------------------------------------------------------------------------------------------------------------
% Basic meta data
%--------------------------------------------------------------------------------------------------------------

\title{Gute Aufgaben in der Mathematik-Hochschullehre}
\author{Prof. Dr. Christian Spannagel}
\date{\today}
\hypersetup{%
    pdfauthor={\theauthor},%
    pdftitle={\thetitle},%
    pdfsubject={Thema Bachelorarbeit},%
    pdfkeywords={bachelor, phhd}
}

%--------------------------------------------------------------------------------------------------------------
% document
%--------------------------------------------------------------------------------------------------------------

\begin{document}

\vspace*{5mm}
\begin{center}
{\Large Thema für eine Bachelorarbeit}
\end{center}

\printtitle
\vspace*{1cm}

In der Mathematikdidaktik für die Grundschule und die Sekundarstufe gibt es zahlreiche Konzepte und Kriterien für gute, d.h. lernwirksame und motivierende, Aufgaben. In der Mathematik-Hochschullehre wurden diese Ideen allerdings noch nicht in der Breite wahrgenommen. Gegebenenfalls lassen sich die Konzepte aber auch nicht direkt auf die Hochschullehre übertragen.  

In der Bachelorarbeit sollten Konzepte und Kriterien für gute Mathematikaufgaben aus der Literatur zusammengestellt und auf die Hochschullehre angewendet werden. Dies erfolgt anhand eines exemplarischen Themas der Lehrveranstaltung \glqq{}Inside Math!\grqq{}. 

Mögliche Tätigkeiten sind:
\begin{itemize}
\item \textbf{Literaturrecherche:} Aufarbeitung der Literatur zu guten Aufgaben im Mathematikunterricht
\item \textbf{Übertragung auf die Hochschullehre:} Prüfung und ggf. Anpassung der Konzepte und Kriterien auf die Mathematiklehre in der Hochschule
\item \textbf{Exemplarische Umsetzung} Anwendung der Konzepte und Kriterien auf Aufgaben eines exemplarischen Teilgebiets der Veranstaltung \glqq{}Inside Math!\grqq{}.
\end{itemize}

\vspace*{10mm}

\printlicense

\printsocials



\end{document}
