\documentclass{../../cssheet}

%--------------------------------------------------------------------------------------------------------------
% Basic meta data
%--------------------------------------------------------------------------------------------------------------

\title{Blended Learning in Mathematik-Lehrveranstaltungen}
\author{Prof. Dr. Christian Spannagel}
\date{\today}
\hypersetup{%
    pdfauthor={\theauthor},%
    pdftitle={\thetitle},%
    pdfsubject={Thema Bachelorarbeit},%
    pdfkeywords={bachelor, phhd}
}

%--------------------------------------------------------------------------------------------------------------
% document
%--------------------------------------------------------------------------------------------------------------

\begin{document}

\vspace*{5mm}
\begin{center}
{\Large Thema für eine Bachelorarbeit}
\end{center}

\printtitle
\vspace*{1cm}

In Lehrveranstaltungen an der Hochschule werden mittlerweile oftmals Präsenz- und Onine-Lernräume miteinander verschränkt (\emph{Blended Learning}). Dabei müssen sich Lehrende die Frage stellen, welche Lernaktivitäten die Studierenden online durchführen sollen und welche im Hörsaal bzw. im Seminarraum. In dieser theoretischen Arbeit soll der Stand der Forschung zu dieser Frage mit Bezug auf das Fach Mathematik ermittelt werden mit dem Ziel, daraus Empfehlungen abzuleiten.

Mögliche Tätigkeiten sind:
\begin{itemize}
\item \textbf{Literaturrecherche:} Aufarbeitung der Literatur zu Blended Learning in der Mathematiklehre
\item \textbf{Analyse:} Erstellung einer Matrix, welche Lernaktivitäten in welcher Umgebung durchgeführt werden können bzw. sollen
\item \textbf{Empfehlungen:} Ableitung von Empfehlungen für die Gestaltung einer Blended-Learning-Veranstaltung im Bereich Mathematik.
\end{itemize}

\vspace*{10mm}

\printlicense

\printsocials



\end{document}
