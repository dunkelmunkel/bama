\documentclass{cssheet}

%--------------------------------------------------------------------------------------------------------------
% Basic meta data
%--------------------------------------------------------------------------------------------------------------

\title{Aktivierung von Vorwissen}
\author{Prof. Dr. Christian Spannagel}
\date{\today}
\hypersetup{%
    pdfauthor={\theauthor},%
    pdftitle={\thetitle},%
    pdfsubject={Thema Bachelorarbeit},%
    pdfkeywords={bachelor, phhd}
}

%--------------------------------------------------------------------------------------------------------------
% document
%--------------------------------------------------------------------------------------------------------------

\begin{document}

\vspace*{5mm}
\begin{center}
{\Large Thema für eine Bachelorarbeit}
\end{center}

\printtitle
\vspace*{1cm}

Im Kontext der Diskussion um guten Unterricht wird immer wieder betont, wie wichtig die Aktivierung von Vorwissen ist. Lehrende müssen dazu analysieren, welches Vorwissen für ein bestimmtes Thema relevant ist, und sie müssen eine entsprechende Aktivierung von Vorwissen gestalten. Im Kontext der Lehrveranstaltung \glqq{}Inside Math!\grqq{} befassen sich die Studierenden mit unterschiedlichen Themenbereichen. Die Studierenden haben oftmals die entsprechenden notwendigen Konzepte aus der Schule aber nicht mehr parat. Im Rahmen der Bachelorarbeit sollen zu einem bestimmten Themenbereich aus der Veranstaltung Materialien zur Aktivierung des notwendigen Vorwissens erstellt werden.

Mögliche Tätigkeiten sind:
\begin{itemize}
\item \textbf{Literaturrecherche:} Aufarbeitung der Literatur zur Aktivierung von Vorwissen
\item \textbf{Analyse:} Didaktische Analyse eines Themenbereichs im Hinblick auf notwendiges Vorwissen
\item \textbf{Entwicklung:} Erstellung von Materialien zur Aktivierung des Vorwissens
\end{itemize}

\vspace*{10mm}

\printlicense

\printsocials



\end{document}
