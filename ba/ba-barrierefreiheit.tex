\documentclass{../cssheet}

%--------------------------------------------------------------------------------------------------------------
% Basic meta data
%--------------------------------------------------------------------------------------------------------------

\title{Barrierefreies Blended Learning in der Mathematik}
\author{Prof. Dr. Christian Spannagel}
\date{\today}
\hypersetup{%
    pdfauthor={\theauthor},%
    pdftitle={\thetitle},%
    pdfsubject={Thema Bachelorarbeit},%
    pdfkeywords={bachelor, phhd}
}

%--------------------------------------------------------------------------------------------------------------
% document
%--------------------------------------------------------------------------------------------------------------

\begin{document}

\vspace*{5mm}
\begin{center}
{\Large Thema für eine Bachelorarbeit}
\end{center}

\printtitle
\vspace*{1cm}

Der Einsatz digitaler Werkzeuge in der Hochschullehre kann Barrieren abbauen, aber auch neue erzeugen. Im Kontext von Blended-Learning-Konzepten muss geprüft wer-den, welche Unterstützungsbedarfe Menschen mit unterschiedlichen Einschränkungen haben. In dieser Arbeit soll der Stand der Forschung zu barrierefreier Hochschullehre im Zusammenhang mit der Nutzung digitaler Werkzeuge zusammengestellt werden. Dabei sollen Empfehlungen für eine Mathe\-ma\-tik-Lehr\-ver\-an\-stal\-tung abgeleitet werden. Im Rahmen dieser Arbeit kann man sich auf einen bestimmten Typ von Unterstützungsbe-darfen fokussieren, etwa für Menschen mit Seh- oder Hörschädigung, mit chronischen Erkrankungen oder mit bestimmten psychischen Erkrankungen.

Mögliche Tätigkeiten sind:
\begin{itemize}
\item \textbf{Literaturrecherche:} Aufarbeitung der Literatur zu barrierefreier digitaler Lehre im Allgemeinen und zu barrierefreier Mathematiklehre im Speziellen, fokussiert auf ei-nen bestimmten Barrierebereich
\item \textbf{Empfehlungen:} Ableitung von Empfehlungen für Lehrende zur Durchführung von barrierefreier digitaler Mathematiklehre
\end{itemize}

\vspace*{10mm}

\printlicense

\printsocials



\end{document}
