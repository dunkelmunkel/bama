\documentclass{cssheet}

%--------------------------------------------------------------------------------------------------------------
% Basic meta data
%--------------------------------------------------------------------------------------------------------------

\title{Feedback zu digitalen Übungsaufgaben}
\author{Prof. Dr. Christian Spannagel}
\date{\today}
\hypersetup{%
    pdfauthor={\theauthor},%
    pdftitle={\thetitle},%
    pdfsubject={Thema Bachelorarbeit},%
    pdfkeywords={bachelor, phhd}
}

%--------------------------------------------------------------------------------------------------------------
% document
%--------------------------------------------------------------------------------------------------------------

\begin{document}

\vspace*{5mm}
\begin{center}
{\Large Thema für eine Bachelorarbeit}
\end{center}

\printtitle
\vspace*{1cm}

Im Kontext der Lehrveranstaltung \glqq{}Inside Math!\grqq{} üben die Studierenden mit Hilfe von interaktiven digitalen Übungsaufgaben. Das verwendete System \textsc{Stack} ermöglicht es zum einen, Übungsaufgaben mit randomisierten Parametern zu stellen, sodass Studierende dieselben Aufgaben mit immer neuen Werten bearbeiten können. Zum anderen bietet das System die Möglichkeit, gezielt Feedback zu geben, beispielsweise zu typischen Fehlern. Im Rahmen der Bachelorarbeit sollen Aufgaben und Feedback zu einem Themenbereich aus der Lehrveranstaltung erstellt werden.

Mögliche Tätigkeiten sind:
\begin{itemize}
\item \textbf{Literaturrecherche:} Aufarbeitung der Literatur zu adaptivem Lernen und Feedback zu interaktiven Übungsaufgaben
\item \textbf{Analyse:} Didaktische Analyse eines Themenbereichs im Hinblick auf die Erstellung von Aufgaben und potenziell mögliches Feedback
\item \textbf{Entwicklung:} Entwicklung der Aufgaben und Feedback-Elemente in \textsc{Stack}
\end{itemize}

\vspace*{10mm}

\printlicense

\printsocials



\end{document}
